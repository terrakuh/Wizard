\chapter{Fazit}
Von den anfänglichen Anforderungen und Ideen wurde tatsächlich ein großer Teil umgesetzt. Es aknn also mit verschiedensten nützlichen UI-Elementen online mit Anderen gespielt werden. Was bisher nicht umgesetzt wurde ist insbesondere eine schöne graphische Oberfläche zur Darstellung des Statistiken. Alle hierfür benötigten Daten werden allerdings in der Datenbank erfasst und können daher in einer späteren Version noch mit einer entsprechenden API Funktion und Darstellung im Frontend umgesetzt werden. Außerdem wurde zwar eine Kalenderfunktion zum Planen gemeinsamer Speiltermine implementiert, allerdings noch nicht vollständig an ein automatisiertes Benachrichtigungssystem angebunden. Zwar sind auch dafür bereits Ansätze von entsprechenden Bot-Implementierungen vorhanden, diese müssten allerding noch fertiggestellt und an den Kalender angebunden werden. Und auch das Feature des teil-automatischen Ausspielens einer Karte ist aktuell nicht implementiert. Allerdings wurden auch, insbesondere im Bereich der GUI einige Features wesentlich detailierter also zu Beginn gefordert ausgearbeitet. Beispiele dafür sind der "Dark-Mode", die Animationen bei der Kartenauswahl oder die stark personalisierbaren Farb- und Benachrichtigungseinstellungen.

Zusammengefasst wurde das Projekt also sehr erfolgreich durchgeführt und hat zu einem sehr brauchbaren Ergebnis geführt. 
Logging
Fokus auf funktionen statt testen
Stuff nicht geschafft
regelmäßige Termine nice
Starke trennung viel kommunikation
zusammen gleiches wissen, aber slow
db von anfang mega

