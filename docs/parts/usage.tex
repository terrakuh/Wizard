\chapter{Nutzung}
Für die Nutzung der Software wird zum einen die Container-Virtualisierungssoftware Docker benötigt, sowie weitere Ressourcen wie Bilder von den Spielkarten und Benachrichtigung-Sounds. Um an den aktuellsten Stand der Software zu gelangen, kann das GitHub Repository mittels Git geklont werden. Hierfür muss folgendes in ein Terminal getippt werden:

\begin{lstlisting}[language=bash]
git clone https://github.com/terrakuh/wizard.git
cd wizard
\end{lstlisting}

Des Weiteren muss die Datei \textit{wizard-assets.zip} im Hauptverzeichnis abgespeichert werden. Die folgende Befehle bauen ein Docker-Image und führen es anschließend aus:

\begin{lstlisting}[language=bash]
sudo docker build . -t wizard
sudo docker run --name wizard -p 8000:8000 wizard
\end{lstlisting}

Nun kann über URL \textit{http://localhost:8080/register} mithilfe des Webbrowsers auf die App zugegriffen werden. Für den ersten Nutzer ist der Token ein leerer String, wohingegen für jeden weiteren Nutzer ein Token manuell in der Datenbank erstellt werden muss. Folgender Befehl fügt einen neuen Token mit dem Wert \textit{token} in die Datenbank hinzu:

\begin{lstlisting}[language=bash]
sudo docker exec -t wizard sqlite3 /data/wizard.db \
	"INSERT INTO register_token(token) VALUES('token')"
\end{lstlisting}

Zusätzlich muss beachtet werden, dass ein Token nur einmalig gleichzeitig in der Datenbank existieren kann.

Die Daten werden standardmäßig in dem Container unter dem Ordner \textit{/data} abgespeichert. Um also einen Datenverlust zwischen verschiedenen Container-Versionen zu vermeiden kann der Container mit einem permanenten Volume gestartet werden:

\begin{lstlisting}[language=bash]
sudo docker volume create wizard-data
sudo docker run --name wizard -p 8000:8000 \
	-v wizard-data:/data wizard
\end{lstlisting}

Außerdem kann die Anwendung, insbesondere zu Testzwecken auch lokal gestartet werden. Dafür müssen zuerst die nötigen Dependencies installiert werden:

\begin{itemize}
	\item Backend (/wizard/backend)
	\begin{lstlisting}[language=bash]
	pip install -r requirements.txt
	\end{lstlisting}
	\item Frontend (/wizard/frontend)
	\begin{lstlisting}[language=bash]
	yarn install
	\end{lstlisting}
\end{itemize}

Ist dies geschehen, können Front- und Backend gestertet werden:

\begin{itemize}
	\item Backend (/wizard/backend)
	\begin{lstlisting}[language=bash]
	python main.py
	\end{lstlisting}
	\item Frontend (/wizard/frontend)
	\begin{lstlisting}[language=bash]
	yarn start
	\end{lstlisting}
\end{itemize}

Damit wird das Backend lokal auf Port 8000 und das Frontend lokal auf Port 8080 gestartet und die Anwendung kann über den Browser erreicht werden.